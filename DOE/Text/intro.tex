%\lipsum[5]

The 2015 Nuclear Science Advisory Committee (NSAC) Long Range Plan
\href{https://science.energy.gov/np/nsac/}
{``Reaching  for the Horizon''} outlined the key fundamental  questions{\it 
\begin{itemize}
	\item [] How are the sea quarks and gluons, and their spins, distributed in
		space and momentum inside the nucleon? 
	\item [ ] How does the nuclear environment affect the distribution of
		quarks and gluons and their interactions in nuclei?
	\item [] Where does the saturation of gluon densities set in?
\end{itemize}}
\noindent 
under one overarching theme \\
\indent {\it  
``How does subatomic matter organize itself and what phenomena emerge?''.  
} \\ \noindent 
The scientific  urge to answer these questions lead the NSAC 
to recommend EIC as the ``highest priority for new facility construction
following the completion of FRIB''. In the mid 2018, this recommendation 
was supported by the National Academy of Science which  concluded 
``that the science questions
regarding the building blocks of matter are compelling and that an EIC is
essential to answering these questions.'' 
This, however, can only be achieved   
with appropriate theoretical support %in the form of providing prediction 
and development of new theoretical tools built upon 
first-principle approaches  to 
the theory of strong interactions. 
This proposal seeks funding for building such a tool 
with a broader application range which includes also high energy hadron-hadron 
collisions and ultra peripheral collisions  of proton-nucleus and nucleus-nucleus.   

Besides the EIC physics, 
the successful implementation of the proposal will have a significant impact on 
understanding the results of other planned or on-going experimental programs  at the large 
facilities in the U.S. (RHIC, CEBAF) and Europe (CERN). 



