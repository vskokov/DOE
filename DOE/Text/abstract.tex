%\begin{document}
\vspace{0.5em}
\noindent
{\bf Title:} \Title

\vspace{0.5em}
\noindent
{\bf Lead Institution:} \Institute

\vspace{0.5em}
\noindent
{\bf Principal Investigator:} \Investigator


\vspace{0.5em}
\noindent


\vspace{0.5em}
\noindent
{\bf Abstract:} 
Scattering processes and soft/semihard particle production in high-energy
hadron-hadron and lepton-hadron collisions are largely defined  by the 
small Bjorken $x$ component of the hadron(s) wave-function. 
These  collisions provide access to kinematic regions 
in nucleons and nuclei where parton densities are extremely high 
and dominated by gluons. In contrast to 
well-understood dilute QCD dynamics at small distance scales, 
high gluon density QCD is an intrinsically new, nonlinear regime
forbidding usual perturbative treatment. Nevertheless due to formation of a semi-hard scale, 
the so-called saturation momentum, the theory is calculable  semi-analytically using 
weak coupling, but non-perturbative techniques.
This proposal aims to develop  a unifying quantitative framework for 
describing  novel aspects of small-$x$ physics in hadron-hadron,
lepton-hadron, and Ultra Peripheral A-A collisions.

This work is especially relevant to experiments at  Relativistic Heavy Ion Collider (BNL),
the Large Hadron Collider (CERN) and a future Electron-Ion Collider (BNL and JLab).  


%{\TODO (should be shortened -- currently 119, the rule: 100 words or less )
%}
% participating in a collision.
%can still
%be utilized to extract   semi-analytic results for most of the observables. 


\vspace{0.5em}
\noindent
{\bf Summary of Proposed Work:}
The main objectives  are 
\begin{itemize}

	%\item Derive complete analytic results for first-saturation correction 
	%	in projectile gluon density for single inclusive and double inclusive 
	%	gluon production in classical approximation for 
	%	asymmetric collisions (e.g. proton-nucleus collisions).    

	%	Most of the semi-analytic calculations for particle production in
	%	the saturation framework are 
	%	either obtained in the dilute-dilute approximation 
	%	(fully neglecting multiple rescattering) or in dilute-dense approximation
	%	(multiple rescattering if fully accounted for in the dense target).

	%	However, to perform a systematical analysis of particle production 
	%	and correlations in hadronic collisions 
	%	it is crucial to compute the first non-trivial correction to 
	%	the dilute-dense limit. This correction is termed the first saturation correction. 

	\item Build a unified framework for a systematic 
		first-principle description of multiparticle production and
		multiparticle correlations at small $x$ in 
		various colliding systems (p-p/A, e-p/A and ultra peripheral p/A-A). The critical steps include:
		\begin{itemize}
			\item Deriving complete analytic results for the first-saturation correction 
		in projectile gluon density for single inclusive and double inclusive 
		gluon production in the classical approximation for 
		asymmetric collisions (e.g. proton-nucleus collisions).
			\item Theoretical  development of a reweighting technique
		enabling an access to the high multiplicity tail of 
		hadronic wave-function/particle production. 
			\item Proper account for  small $x$ evolution in order to restore the
		three-dimensional snapshot of multiparticle production in collisions
		and to elicit the dependence on the collision energy.
			\item Making the source code(s) performing numerical lattice 
				simulations for multiparticle production publicly available. 
		\end{itemize}

		{\it Potential Impact:} 
		These first-principle quantitative studies of the initial state effects in hadronic collisions 
		may lead to a potential shift of paradigm in our understanding of 
		particle production and, specifically,  the origin of collectivity in hadronic collisins. Additionally, 
		on this stage of 
		the EIC R\&D program, quantitative  predictions for the observables sensitive to 
		gluon saturation or to various gluon distribution functions  
		(e.g. the linearly polarized gluon distribution in an unpolarized hadron)  
		in the relevant energy range 
		may influence yet flexible design of EIC detectors by providing a
		optimal kinematic range for performing corresponding measurements.  

	\item Quantitatively study momentum entanglement in hadronic wave function. Derive 
		small $x$ evolution for the full density matrix including the off-diagonal components 
		in the basis of valence charge density. 
		Extract small $x$ evolution of the associated entanglement entropy. Identify potential 
		experimental observables. 

		{\it Potential impact:} Quantum entanglement is a universal phenomenon underlying 
		the behavior of quantum systems of diverse nature. The concept
		does not rely on small coupling methods and thus may facilitate extension of 
		evolution equations for entanglement entropy to an arbitrary value of the strong coupling.     
		

	\item Provide a cohesive picture of the role of quantum statistics at small $x$. 
		Study associated effects in gluon and quark production. Compute the initial state
		background contribution to the Chiral Magnetic Effect (CME). 
		
		{\it Potential impact:} Experimental data collected by CMS (LHC) collaboration in p-A collisions
		challenges the conventional picture of the flow-driven CME background established in A-A
		collisions. The initial state correlations may explain the observed discrepancy.     


\end{itemize}

\vspace{0.5em}
\noindent
{\bf List of Personal:}
\begin{enumerate}
	\item Faculty: Vladimir Skokov (PI)
	\item Postdoctoral Fellows: to be hired 
	\item Graduate students:  Haowu Duan,  Gregory Johson 
\end{enumerate}

