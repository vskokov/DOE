The two decades worth of experimental
measurements at RHIC and, then, the LHC 
have provided many unexpected results, including 
a strong evidence for the formation of 
a strongly coupled plasma of quarks and gluons in
heavy-ion collisions at high energy~\cite{Shuryak:2003xe,Shuryak:2004cy,Adams:2005dq,Song:2010mg}. 
This plasma  
demonstrated properties of a nearly perfect fluid; 
this fact facilitated  theoretical description 
of the collisions dynamics starting just  about 1 fm/c after the heavy-ion impact 
in the framework of hydrodynamics~\cite{Schafer:2009dj,Song:2010mg,Romatschke:2017ejr}.   

The success of the hydrodynamic description, however, cannot be complete 
without a detailed understanding of the initial 
non-equilibrium state. The properties of this state go beyong the range of applicability 
of hydrodynamics and  are little known; 
the evolution of this state towards equilibrated 
thermal nearly perfect liquid  is not well understood. 
One dominant mechanism describing the initial phase is 
based on the saturation framework, also widely known as 
the Color Glass Condensate~\cite{Iancu:2002xk,Albacete:2014fwa,KovchegovLevin}. According to the framework, the 
high energy particle production and scattering processes are 
dominated by the classical gluon fields providing a 
background for systematic weak-coupling 
computation of quantum correction on top of it.  

In laboratory, 
collisions of heavy-ions create probably the most optimal envi\-ronment for 
probing quark-gluon plasma near equilibrium, but 
at the same time they are poorly suited to study the 
initial state particle production. This is because 
most of the observables in heavy-ion collisions are 
sensitive not only to  initial state but also 
to  final sate interactions. However, to uniquely map the transport properties 
of the plasma, it is critical to extract information 
about the initial state in collisions where 
the final state is better understood and the 
initial state is expected to play the dominant role.
This necessitates to probe a nucleus  and a nucleon with the smallest projectiles: 
proton and ultimately electron. 
Theoretically, a controlled,  first principle description of such collisions (p-p, p-A, e-A) 
is not as complex as A-A collisions 
and at high energy can be performed in a common quantitative framework: the saturation framework. 
Later on the results of the framework may be transferred to describe the initial state of A-A collisions. 


One example of similar strategy was realized recently in Refs.~\cite{Mantysaari:2016ykx,Mantysaari:2016jaz,Mantysaari:2017cni} 
where the saturation motivated model (IP-SAT, see Ref.~\cite{Kowalski:2003hm}) for e-p collisions  was used to constrain the 
shape fluctuations of the proton wave function by studying 
coherent and incoherent diffractive 
vector meson production at HERA. The proton shape fluctuations may have direct 
impact on properties of the matter created in A-A collisions from 
the initial state in peripheral to ultra central collisions. 



Theoretical analysis of future experimental data from an Electron-Ion Collider  will 
definitely make greater impact on our understanding of the dynamics in 
p-p, p-A and A-A collisions.  % and the initial state of A-A collisions. 
The reverse is also partially true as the data collected 
in p-p and p-A collisions currently drive an active 
development of the first principle approaches to 
high energy QCD physics. Additionally p-p and p-A 
collisions allow us to probe a different phase space 
of high energy  QCD, complementing future EIC coverage, and thus 
better constraining the applicability range of our frameworks. 

Before an EIC comes into operation, ultra peripheral collisions (UPC) 
of p-A and A-A provide a unique opportunity to 
further sharpen theoretical methods. It is clear that 
many smoking-gun signals of new QCD dynamics at high energy 
will be measured with rather high statistics 
by collecting UPC data at the LHC and RHIC. Additionally, researchers at the LHC 
seriously consider 
future physics opportunities for studying high-density QCD 
with UPC in ions and proton beams. This is why it is 
crucial to have a common approach to a wide energy or Bjorken $x$ range; 
in saturation framework the energy dependence  is captured by the small-$x$ 
renromalization group evolution equations.  
A modern pinnacle of performing numerical analysis of the small-$x$ evolution is based on 
the leading order  JIMWLK (Jalilian-Marian--Iancu--McLerran--Weigert--Leonidov--Kovner) equation~\cite{JalilianMarian:1997dw, 
JalilianMarian:1997gr,
Iancu:2001ad, 
Iancu:2000hn}  which resums powers of $\alpha_s \log x^{-1}$
and extends the applicability range of BK (Balitsky-Kovchev) equation \cite{Balitsky:1995ub,Balitsky:1998ya,Kovchegov:1999yj,Kovchegov:1999ua} to finite number of colors $N_c$. 
It has not yet be fully utilized to describe gluon production, 
while it paves a way to reducing model dependence towards  
a first-principle framework.  

%Future precision phenomenology requires numerical solution of the 
%next-to-leadin order JIMWLK, which was recently derived. 
%It however still awaits \ldots 

The number of unresolved theoretical issues complemented by a number of 
quite puzzling and unexplained features of experimental data (see below and  Secs.~\ref{sec:p1},~\ref{sec:p2} and~\ref{sec:p3})  
and the need for the development of new theoreticall tools to EIC physics 
require to solidify our understanding of small $x$ dynamics, 
to develop first-principle based event generators and, finally, to  
explore new approaches and ideas. 


%Developping a first-principle theoretical model is required to 

Cristallizing %ummarizing 
the above, there are four  major motivational themes which  drive our interest  

\vspace{0.5em}

\noindent
{\bf Theme 1: 
To differentiate initial state effects from  final state dynamics of strongly coupled quark-gluon plasma. 
}
Multiparticle production and correlations are cannonical measurements for a) the formation of the strongly coupled 
plasma, see e.g. Refs.~\cite{Schafer:2009dj,Song:2010mg}; b) the Chiral Magnetic Effect~\cite{Kharzeev:2007jp}, the novel transport mechanism due to QCD quantum axial anomaly;
%c) the nuclear modification factor at given ``centrality'' (defined as the activity of an event) and 
c) higher order cumulants of proton fluctuations as a signature of a critical point or first-order phase transition~\cite{Stephanov:2008qz}. 
For all of these items in the list, it is important to disentangle the initial state contribution from 
the genuine final state effects in near-equilibrium quark-gluon plasma. Quantifying the role of the initial state 
will further strengthen the discovery potential for all these key measurements and improve the chances of 
establishing these novel phenomena.   

\vspace{0.5em}

\noindent
{\bf Theme 2: 
	To find  a better, simpler description of complex problems in hadron structure and 
high energy QCD. 
}	 
At small $x$ or at high energy, the number of gluons grows rapidly; intuitively, it is  
reasonable to expect that an effective, statistical description of some if not all 
observables can be feasible and accurate. There are already known examples where 
statistical analogy  was implemented and offered fresh insights on complex problems: 
-- treatment of of high-energy evolution as a reaction-diffusion process in statistical physics~\cite{Munier:2003vc,Iancu:2004es,Kutak:2011rb}, 
-- parametrization of parton distribution functions motivated by quantum statistics distributions~\cite{Mac:1989ki,Bhalerao:1996fc,Bourrely:2001du}, 
-- establishing the dominance of Bose enhancement in high multiplicity events~\cite{Kovner:2018azs}, 
-- entropy of small $x$ gluons in hadronic wave function and its relation to 
multiplicity and entropy of produced hadrons in the final state~\cite{Peschanski:2012cw,Kovner:2015hga,Kharzeev:2017qzs,Shuryak:2017phz,Hagiwara:2017uaz}.  


Along these lines, one may hypothesize that small $x$ evolution may be generalized beyond 
weakly coupled regime through the universal notion of the entanglement entropy. 
It could be also possible  that strict bounds on the entanglement entropy~\cite{Holzhey:1994we,Calabrese:2005zw}
may lead 
to a resolution of the QCD unitarity problem at high energy. 

Besides, the statistical description, recently established set of dualities between the vertices 
of the infra-red triangle~\cite{Strominger:2017zoo,Pate:2017vwa,Ball:2018prg} may also bring better understanding of particle production in near classical regime. 

\vspace{0.5em}



\noindent

{\bf Theme 3:  Describe puzzling features of the data. }
There are puzzling features of the data that may potentially be described 
with the saturation/CGC formalism; these includes: 
-- long-range rapidity correlations and apparent collectivity in p-p and p-A 
collisions~\cite{Khachatryan:2010nk,Aad:2010ac,Aaij:2014pza,ALICE:2017pcy},  
-- baryon-baryon correlation in p-p collisions~\cite{Adam:2016iwf}, 
-- re-emergence of the Cronin peak in multiplicity biased p-A collisions at high energy~\cite{ALICE:2012mj} , 
-- strong correlations between soft-hard particle production in p-p and p-A collisions~\cite{ATLAS:2014cpa,Aad:2015ziq}. 


\vspace{0.5em}

\noindent

{\bf Theme 4:  Rigorous theoretical predictions 
on gluon saturation dynamics in experimental observables at a future EIC.}
An EIC has a strong discovery potential 
for a novel QCD phase. Theoretical development and in particular 
predictions based on saturation dynamics are required to claim  the discovery. 

\vspace{2em}

The success of this project is defined by delivering the following key results 
\begin{itemize}
    \item Developing a systematic framework for particle production at 
		small $x$. Making the associated source code(s) performing numerical simulations available online. 
		%for 
		%the developed framework available online. 
		This item is an integral part 
		of the proposal; it was pointed out on multiple occasions that 
		there are no publicly available small $x$ codes or they are not well 
		documented. 
		%With our experience of publishing
		%the simulation code MCDijet, we    
    \item Derivation of the 
		small-$x$ evolution equations for the off-diagonal components 
		of the density matrix in the basis of the valence charge density.
		Derivation of the small-$x$ evolution equations for the momentum 
		entanglement entropy and seeking for universal features allowing to 
		a potential extension to arbitrary coupling.  
    \item Providing a detailed and cohesive picture of the role of 
		quantum statistics at small $x$. Ultimately establishing 
		connection to 
		the Kulish-Fadeev formalism, the color memory effect, and 
		BMS (Bondi-Metzner-Sachs) symmetry and their consequences at small $x$.   
		Publishing a detailed review on the role of quantum statistics 
		crystallizing synergy of the results obtained by the PI and 
		other researchers. 
\end{itemize}
